\documentclass{report}

\usepackage{booktabs}
\usepackage{float}
\usepackage{comment}
\usepackage{graphicx} 
\usepackage{amsthm} 
\usepackage{amsfonts}
\usepackage{pdfpages} 
\usepackage{siunitx} 
\usepackage{mhchem}
\usepackage{subfloat}
\usepackage{fullpage}
\usepackage{array}
\usepackage[toc,page]{appendix}

\newcommand{\ecoli}{\textit{E. coli}}
\newcommand{\ecloni}{\textit{E. cloni 10g}} 
\newcommand{\bsub}{\textit{B. subtilis}} 
\newcommand{\pbad}{P$_{BAD}$} 
\newcommand{\plux}{P$_{LUX}$}
\newcommand{\etal}{\textit{et al.}} 
\newcommand{\invivo}{\textit{in vivo}}
\newcommand{\invitro}{\textit{in vitro}} 
\newcommand{\csix}{3OC6HSL}

\newcommand{\tred}[1]{\textcolor{red}{#1}}
\newcommand{\tgr}[1]{\textcolor{green}{#1}}

% Units
\DeclareSIUnit{\molar}{M} 
\newcommand{\ul}{\si{\micro\litre}} 
\newcommand{\um}{\si{\micro\metre}}
\newcommand{\ml}{\si{\milli\litre}} 
\newcommand{\nm}{\si{\nano\metre}} 
\newcommand{\ugml}{\si{\micro\gram\per\milli\litre}} 
\newcommand{\degC}{\si{\degreeCelsius}}

% Misc
\newcommand{\tenx}{\ensuremath{10\times}}
\newcommand{\fortyx}{\ensuremath{40\times}}
\newcommand{\sixtyx}{\ensuremath{60\times}}
\newcommand{\hundredx}{\ensuremath{100\times}}
\newcommand{\odsix}{\ensuremath{\textrm{OD}_{600}}}
\newcommand{\Pspo}{\ensuremath{\textrm{P}_{\textrm{\scriptsize SPO1-26}}}}

% Matrix/ vector shit
\let\oldhat\hat 
\renewcommand{\vec}[1]{\mathbf{#1}}
\renewcommand{\hat}[1]{\oldhat{\mathbf{#1}}}
\let\oldcaption\caption
%\renewcommand{\caption}[2][]{\oldcaption[#1]{\small\textbf{#1.} #2\normalsize}}
\renewcommand{\caption}[2][]{\oldcaption[#1]{\textbf{#1.} #2}}

\newcommand{\mat}{\mathbf} 
\newcommand{\invmat}[1]{\mat{#1}^{-1}}
\newcommand{\deltap}{\Delta \vec{p}} 
\newcommand{\deltaL}{\Delta \vec{L}}
\newcommand{\deltag}{\Delta g} 
\newcommand{\Iinv}{\mat{I}^{-1}}
\newcommand{\Minv}{\mat{M}^{-1}}
\newcommand{\crossmat}[1]{{\left[#1\right]}_{\times}}

\newtheorem{remark}{Remark} 
\newtheorem*{note}{Note} 

\begin{document} 

\chapter{Eigenmorphs? Rational design of pattern and morphology in eigenspace}

\section{Background maths}
\section{Eigenfunctions of the laplacian}
Eigenfunctions of the laplacian $\nabla^2$ satisfy
\begin{equation}
\nabla^2 v(\vec{x}) + k^2 v(\vec{x}) = 0
\end{equation}

Given some boundary condition, such as $v=0$ or $\nabla v=0$ at the boundary of the
domain there is a discrete (infinite) set of functions $v_i(\vec{x})$, with
eigenvalue $\lambda_i$.  For a 1-dimentionsal domain with zero boundary
conditions for example, the eignfunctions are sine waves $sin(k\pi/L)$ where
$L$ is the length of the domain and $k \in {0,1,2,3...}$.

On other arbitrary domains we have to compute them numerically. You can see they
are waves because the equation says when $v>0$ the curvature $\nabla^2v<0$ and
vice versa.

\section{Eigenfunction expansion or projection}
They also have the nice property that they form a complete basis for the space
of differentiable functions that satisfy the boundary conditions. That means that
any pattern (gene expression, signal concentration, etc.) on a given domain
(colony, organism, etc.) can be represented as a series
\begin{equation}
f(\vec{x}) = \sum_{i} w_i v_i(\vec{x})
\end{equation}
Which is exactly the Fourier series for a rectangular or linear domain.

The equivalent of the Fourier transform for arbitrary shaped domains is the
eigenspectrum, which gives the amount of each eigenfunction represented in a
particular pattern
\begin{equation}
w_i = \langle f, v_i \rangle = 
\end{equation}

\section{Eigenvectors and discrete space}
For computation we represent patterns on a regular grid like an image. It turns
out convenient to string this out as a 1-dimensional vector of length $w\times
h$ for an image of width $w$ and height $h$. The operator $\nabla^2$ is then a
matrix $\mat{M}$ which has eigenvectors $\vec{v}_i$ such that $\mat{A}\vec{v}_i
= \lambda_i \vec{v}_i$. Note that now there are a finite number of eigenvectors
equal to the number of grid points or pixels in the image.

The eigenspectrum is just a basis transformation from the grid basis, where each
value in the grid (image) is an element of the vector, to the basis of
eigenvectors. This is the natural frequency domain of the shape defined by the
boundary conditions. Lets call this the eigenspace.

In discrete space we can define a matrix of eigenvectors
$\mat{W}$ where each column is an eigenvector $\vec{v}_i$ such that
\begin{equation}
\vec{w} = \mat{W}\vec{g}
\end{equation}
is the eigenspace transform of the image $\vec{g}$, and
\begin{equation}
\vec{g} = \mat{W}^T\vec{w}
\end{equation}
is the inverse transform. This is because the eigenvectors form an orthonormal
basis for the space of images (patterns).

\section{Design of pattern in eigenspace}
This means we can compute the eigenspectrum of any pattern, for example one that
we might want to design a genetic system to produce. Alternatively we could
design the spectrum of shape that we require from our genetic system. It would
be easy (relatively) to make a tool that allows you to adjust a curve and see
the resulting pattern (in Matlab for example). This curve could be a polynomial
for example. It seems reasonable that a useful biological pattern should be
localised in eigenspace (frequency space). If all wavelengths are represented
equally we have white noise, which possibly is maximal entropy. 

What we would like to do is choose a pattern and infer a genetic mechanism
coupled to signalling that could generate it. These genetic components would
come from a registry of characterised parts. For theoretical work we show the
principle for example by randomly generating part parameters to make a library.

\section{Reacion diffusion systems}
In this sense any patterning system can be seen as selecting (weighting)
some eigenvectors more than others. Reaction-diffusion systems are an obvious
example (there are others, see Murray). 
\begin{eqnarray}
u_t &=& \nabla^2u + \gamma f(u,v) \\
v_t &=& d \nabla^2v + \gamma g(u,v)
\end{eqnarray}

In general we can't predict the final pattern
without solving numerically, but close to an equilibrium (homogeneous or
heterogeneous) we can approximate the reaction term as linear in the
perturbation $\vec{w} = (u-u_0,v-v_0)^T$ where $(u_0,v_0)$ is the equilibrium
giving
\begin{equation}
\vec{w}_t = \nabla^2\vec{w} + \mat{A}\vec{w}
\end{equation}
where $\mat{A}$ is the Jacobian
\begin{equation}
\mat{A}=
 \left[
  \begin{array}{ c c }
     f_u & f_v \\
     g_u & g_v
  \end{array} \right]
\end{equation}

We can use this to do a linear stability analysis and test for patterning. The
definition of a Turing system is that it has an unstable homogeneous
equilibrium, and a stable heterogeneous equilibrium. Around the homogeneous
equilibrium the behaviour can be approximated by the equation above. Expanding
the solution in eigenvectors we get
\begin{equation}


\section{Morphology }

\section{}
\end{document}

